%%%%%%%%%%%%%%%%%
% This is an sample CV template created using altacv.cls
% (v1.7, 9 August 2023) written by LianTze Lim (liantze@gmail.com). Compiles with pdfLaTeX, XeLaTeX and LuaLaTeX.
%
%% It may be distributed and/or modified under the
%% conditions of the LaTeX Project Public License, either version 1.3
%% of this license or (at your option) any later version.
%% The latest version of this license is in
%%    http://www.latex-project.org/lppl.txt
%% and version 1.3 or later is part of all distributions of LaTeX
%% version 2003/12/01 or later.
%%%%%%%%%%%%%%%%

%% Use the "normalphoto" option if you want a normal photo instead of cropped to a circle
% \documentclass[10pt,a4paper,normalphoto]{altacv}

\documentclass[10pt,a4paper,ragged2e,withhyper]{altacv}
%% AltaCV uses the fontawesome5 and packages.
%% See http://texdoc.net/pkg/fontawesome5 for full list of symbols.

% Change the page layout if you need to
\geometry{left=1.25cm,right=1.25cm,top=1.5cm,bottom=1.5cm,columnsep=1.2cm}

% The paracol package lets you typeset columns of text in parallel
\usepackage{paracol}

% Change the font if you want to, depending on whether
% you're using pdflatex or xelatex/lualatex
% WHEN COMPILING WITH XELATEX PLEASE USE
% xelatex -shell-escape -output-driver="xdvipdfmx -z 0" sample.tex
\ifxetexorluatex
  % If using xelatex or lualatex:
  \setmainfont{Roboto Slab}
  \setsansfont{Lato}
  \renewcommand{\familydefault}{\sfdefault}
\else
  % If using pdflatex:
  \usepackage[rm]{roboto}
  \usepackage[defaultsans]{lato}
  % \usepackage{sourcesanspro}
  \renewcommand{\familydefault}{\sfdefault}
\fi

% Change the colours if you want to
\definecolor{SlateGrey}{HTML}{2E2E2E}
\definecolor{LightGrey}{HTML}{666666}
\definecolor{DarkPastelRed}{HTML}{450808}
\definecolor{PastelRed}{HTML}{8F0D0D}
\definecolor{GoldenEarth}{HTML}{E7D192}
\colorlet{name}{black}
\colorlet{tagline}{PastelRed}
\colorlet{heading}{DarkPastelRed}
\colorlet{headingrule}{GoldenEarth}
\colorlet{subheading}{PastelRed}
\colorlet{accent}{PastelRed}
\colorlet{emphasis}{SlateGrey}
\colorlet{body}{LightGrey}

% Change some fonts, if necessary
\renewcommand{\namefont}{\Huge\rmfamily\bfseries}
\renewcommand{\personalinfofont}{\footnotesize}
\renewcommand{\cvsectionfont}{\LARGE\rmfamily\bfseries}
\renewcommand{\cvsubsectionfont}{\large\bfseries}


% Change the bullets for itemize and rating marker
% for \cvskill if you want to
\renewcommand{\cvItemMarker}{{\small\textbullet}}
\renewcommand{\cvRatingMarker}{\faCircle}
% ...and the markers for the date/location for \cvevent
% \renewcommand{\cvDateMarker}{\faCalendar*[regular]}
% \renewcommand{\cvLocationMarker}{\faMapMarker*}


% If your CV/résumé is in a language other than English,
% then you probably want to change these so that when you
% copy-paste from the PDF or run pdftotext, the location
% and date marker icons for \cvevent will paste as correct
% translations. For example Spanish:
% \renewcommand{\locationname}{Ubicación}
% \renewcommand{\datename}{Fecha}


%% Use (and optionally edit if necessary) this .tex if you
%% want to use an author-year reference style like APA(6)
%% for your publication list
% \input{pubs-authoryear.tex}

%% Use (and optionally edit if necessary) this .tex if you
%% want an originally numerical reference style like IEEE
%% for your publication list
\input{pubs-num.tex}

%% sample.bib contains your publications
\addbibresource{sample.bib}

\begin{document}
\name{Victor Pereira Lima}
\tagline{Desenvolvedor de software}
%% You can add multiple photos on the left or right
\photoR{2.8cm}{perfil}
% \photoL{2.5cm}{Yacht_High,Suitcase_High}

\personalinfo{%
  % Not all of these are required!
  \email{lima.p.victor@outlook.com}
  \phone{(11) 97032-6687}
  \location{Barueri, SP}
  \linkedin{vplima}
  \github{limapvictor}
  %% You can add your own arbitrary detail with
  %% \printinfo{symbol}{detail}[optional hyperlink prefix]
  % \printinfo{\faPaw}{Hey ho!}[https://example.com/]

  %% Or you can declare your own field with
  %% \NewInfoFiled{fieldname}{symbol}[optional hyperlink prefix] and use it:
  % \NewInfoField{gitlab}{\faGitlab}[https://gitlab.com/]
  % \gitlab{your_id}
  %%
  %% For services and platforms like Mastodon where there isn't a
  %% straightforward relation between the user ID/nickname and the hyperlink,
  %% you can use \printinfo directly e.g.
  % \printinfo{\faMastodon}{@username@instace}[https://instance.url/@username]
  %% But if you absolutely want to create new dedicated info fields for
  %% such platforms, then use \NewInfoField* with a star:
  % \NewInfoField*{mastodon}{\faMastodon}
  %% then you can use \mastodon, with TWO arguments where the 2nd argument is
  %% the full hyperlink.
  % \mastodon{@username@instance}{https://instance.url/@username}
}

\makecvheader
%% Depending on your tastes, you may want to make fonts of itemize environments slightly smaller
% \AtBeginEnvironment{itemize}{\small}

%% Set the left/right column width ratio to 6:4.
\columnratio{0.6}

% Start a 2-column paracol. Both the left and right columns will automatically
% break across pages if things get too long.
\begin{paracol}{2}
\cvsection{Experiência}

\cvevent{Desenvolvedor de software Junior}{Shape Digital}{Desde de JUL 2023}{São Paulo, SP}
\begin{itemize}
\item Desenvolvimento (implantação de novas funcionalidades, correções de bugs e refatorações de código) da API backend do DBMS, um software de gerenciamento de barreiras de segurança em navios (Python)
\end{itemize}

\divider

\cvevent{Desenvolvedor de software Junior}{MegaWhat Energy}{JAN 2021 -- JAN 2023}{São Paulo, SP}
\begin{itemize}
\item Desenvolvimento da plataforma web da companhia (CMS e API backend em PHP, frontend em Vue.js)
\item Gerenciamento e desenvolvimento de um projeto de dashboard de dados na plataforma
\item Web Scraping de dados do setor de energia (Java)
\item Desenvolvimento de um agendador de scripts do setor de meteorologia (Python)
\item Desenvolvimento do aplicativo da plataforma (Flutter)
\end{itemize}

\divider

\cvevent{Estagiário em desenvolvimento de software}{Aveva Group PLC}{FEV 2020 -- JAN 2021}{São Paulo, SP}
\begin{itemize}
\item Desenvolvimento do produto Aveva EDGE (C++ e C\#)
\item Testes não-automatizados das entregas da equipe
\end{itemize}

\cvsection{Educação}

\cvevent{Mestrado em Ciência da Computação}{Universidade de São Paulo}{Desde de AGO 2023}{}

\divider

\cvevent{Bacharelado em Ciência da Computação}{Universidade de São Paulo}{FEV 2018 -- JUL 2022}{}
Trilhas de especialização completas: Sistemas e Inteligência Artificial

\switchcolumn

\cvsection{Sobre mim}

\begin{quote}
``Entusiasta apaixonado por games e tecnologia em geral, acredito que cada vez que nos dedicamos à programação, temos a oportunidade de aprimorar significativamente a vida das pessoas. Busco constantemente impactar positivamente, seja resolvendo desafios, criando experiências de entretenimento cativantes ou simplificando o trabalho para os próximos programadores que interagirão com o código desenvolvido.''
\end{quote}

\cvsection{Competências}

\begin{itemize}
\item Desenvolvimento de código seguindo boas práticas e padrões de projeto
\item Experiência com testes automatizados
\item Experiência com métodos ágeis
\item Experiência com bancos de dados relacionais (MySQL e SQLServer) e não-relacionais (Redis e MongoDB)
\item Noções de infraestrutura (Linux, Docker, AWS)
\end{itemize}

\cvsection{Idiomas}

\cvskill{Inglês avançado}{4}
\divider

\cvskill{Espanhol básico}{1.5}

%%\cvsection{Most Proud of}

%%\cvachievement{\faTrophy}{Fantastic Achievement}{and some details about it}

%%\cvsection{Strengths}

%%\cvtag{Hard-working}
%%\cvtag{Eye for detail}\\
%%\cvtag{Motivator \& Leader}

%%\cvsection{Publications}

%% Specify your last name(s) and first name(s) as given in the .bib to automatically bold your own name in the publications list.
%% One caveat: You need to write \bibnamedelima where there's a space in your name for this to work properly; or write \bibnamedelimi if you use initials in the .bib
%% You can specify multiple names, especially if you have changed your name or if you need to highlight multiple authors.
%%\mynames{Lim/Lian\bibnamedelima Tze,
%%  Wong/Lian\bibnamedelima Tze,
%%  Lim/Tracy,
%%  Lim/L.\bibnamedelimi T.}
%% MAKE SURE THERE IS NO SPACE AFTER THE FINAL NAME IN YOUR \mynames LIST

%%\nocite{*}

%%\printbibliography[heading=pubtype,title={\printinfo{\faBook}{Books}},type=book]

%%\cvsection{Publications}

%% Specify your last name(s) and first name(s) as given in the .bib to automatically bold your own name in the publications list.
%% One caveat: You need to write \bibnamedelima where there's a space in your name for this to work properly; or write \bibnamedelimi if you use initials in the .bib
%% You can specify multiple names, especially if you have changed your name or if you need to highlight multiple authors.
%%\mynames{Lim/Lian\bibnamedelima Tze,
%%  Wong/Lian\bibnamedelima Tze,
%%  Lim/Tracy,
%%  Lim/L.\bibnamedelimi T.}
%% MAKE SURE THERE IS NO SPACE AFTER THE FINAL NAME IN YOUR \mynames LIST

%%\nocite{*}

%%\printbibliography[heading=pubtype,title={\printinfo{\faBook}{Books}},type=book]

%% Yeah I didn't spend too much time making all the
%% spacing consistent... sorry. Use \smallskip, \medskip,
%% \bigskip, \vspace etc to make adjustments.
\medskip


\end{paracol}


\end{document}
