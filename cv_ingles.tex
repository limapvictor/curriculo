%%%%%%%%%%%%%%%%%
% This is an sample CV template created using altacv.cls
% (v1.7, 9 August 2023) written by LianTze Lim (liantze@gmail.com). Compiles with pdfLaTeX, XeLaTeX and LuaLaTeX.
%
%% It may be distributed and/or modified under the
%% conditions of the LaTeX Project Public License, either version 1.3
%% of this license or (at your option) any later version.
%% The latest version of this license is in
%%    http://www.latex-project.org/lppl.txt
%% and version 1.3 or later is part of all distributions of LaTeX
%% version 2003/12/01 or later.
%%%%%%%%%%%%%%%%

%% Use the "normalphoto" option if you want a normal photo instead of cropped to a circle
% \documentclass[10pt,a4paper,normalphoto]{altacv}

\documentclass[10pt,a4paper,ragged2e,withhyper]{altacv}
%% AltaCV uses the fontawesome5 and packages.
%% See http://texdoc.net/pkg/fontawesome5 for full list of symbols.

% Change the page layout if you need to
\geometry{left=1.25cm,right=1.25cm,top=1.5cm,bottom=1.5cm,columnsep=1.2cm}

% The paracol package lets you typeset columns of text in parallel
\usepackage{paracol}

% Change the font if you want to, depending on whether
% you're using pdflatex or xelatex/lualatex
% WHEN COMPILING WITH XELATEX PLEASE USE
% xelatex -shell-escape -output-driver="xdvipdfmx -z 0" sample.tex
\ifxetexorluatex
  % If using xelatex or lualatex:
  \setmainfont{Roboto Slab}
  \setsansfont{Lato}
  \renewcommand{\familydefault}{\sfdefault}
\else
  % If using pdflatex:
  \usepackage[rm]{roboto}
  \usepackage[defaultsans]{lato}
  % \usepackage{sourcesanspro}
  \renewcommand{\familydefault}{\sfdefault}
\fi

% Change the colours if you want to
\definecolor{SlateGrey}{HTML}{2E2E2E}
\definecolor{LightGrey}{HTML}{666666}
\definecolor{DarkPastelRed}{HTML}{450808}
\definecolor{PastelRed}{HTML}{8F0D0D}
\definecolor{GoldenEarth}{HTML}{E7D192}
\colorlet{name}{black}
\colorlet{tagline}{PastelRed}
\colorlet{heading}{DarkPastelRed}
\colorlet{headingrule}{GoldenEarth}
\colorlet{subheading}{PastelRed}
\colorlet{accent}{PastelRed}
\colorlet{emphasis}{SlateGrey}
\colorlet{body}{LightGrey}

% Change some fonts, if necessary
\renewcommand{\namefont}{\Huge\rmfamily\bfseries}
\renewcommand{\personalinfofont}{\footnotesize}
\renewcommand{\cvsectionfont}{\LARGE\rmfamily\bfseries}
\renewcommand{\cvsubsectionfont}{\large\bfseries}


% Change the bullets for itemize and rating marker
% for \cvskill if you want to
\renewcommand{\cvItemMarker}{{\small\textbullet}}
\renewcommand{\cvRatingMarker}{\faCircle}
% ...and the markers for the date/location for \cvevent
% \renewcommand{\cvDateMarker}{\faCalendar*[regular]}
% \renewcommand{\cvLocationMarker}{\faMapMarker*}


% If your CV/résumé is in a language other than English,
% then you probably want to change these so that when you
% copy-paste from the PDF or run pdftotext, the location
% and date marker icons for \cvevent will paste as correct
% translations. For example Spanish:
% \renewcommand{\locationname}{Ubicación}
% \renewcommand{\datename}{Fecha}


%% Use (and optionally edit if necessary) this .tex if you
%% want to use an author-year reference style like APA(6)
%% for your publication list
% % When using APA6 if you need more author names to be listed
% because you're e.g. the 12th author, add apamaxprtauth=12
\usepackage[backend=biber,style=apa6,sorting=ydnt]{biblatex}
\defbibheading{pubtype}{\cvsubsection{#1}}
\renewcommand{\bibsetup}{\vspace*{-\baselineskip}}
\AtEveryBibitem{%
  \makebox[\bibhang][l]{\itemmarker}%
  \iffieldundef{doi}{}{\clearfield{url}}%
}
\setlength{\bibitemsep}{0.25\baselineskip}
\setlength{\bibhang}{1.25em}


%% Use (and optionally edit if necessary) this .tex if you
%% want an originally numerical reference style like IEEE
%% for your publication list
\usepackage[backend=biber,style=ieee,sorting=ydnt,defernumbers=true]{biblatex}
%% For removing numbering entirely when using a numeric style
\setlength{\bibhang}{1.25em}
\DeclareFieldFormat{labelnumberwidth}{\makebox[\bibhang][l]{\itemmarker}}
\setlength{\biblabelsep}{0pt}
\defbibheading{pubtype}{\cvsubsection{#1}}
\renewcommand{\bibsetup}{\vspace*{-\baselineskip}}
\AtEveryBibitem{%
  \iffieldundef{doi}{}{\clearfield{url}}%
}


%% sample.bib contains your publications
\addbibresource{sample.bib}

\begin{document}
\name{Victor Pereira Lima}
\tagline{Software Developer}
%% You can add multiple photos on the left or right
\photoR{2.8cm}{perfil}
% \photoL{2.5cm}{Yacht_High,Suitcase_High}

\personalinfo{%
  % Not all of these are required!
  \email{lima.p.victor@outlook.com}
  \phone{+ 55 11 97032-6687}
  \location{Barueri, SP, Brazil}
  \linkedin{vplima}
  \github{limapvictor}
  %% You can add your own arbitrary detail with
  %% \printinfo{symbol}{detail}[optional hyperlink prefix]
  % \printinfo{\faPaw}{Hey ho!}[https://example.com/]

  %% Or you can declare your own field with
  %% \NewInfoFiled{fieldname}{symbol}[optional hyperlink prefix] and use it:
  % \NewInfoField{gitlab}{\faGitlab}[https://gitlab.com/]
  % \gitlab{your_id}
  %%
  %% For services and platforms like Mastodon where there isn't a
  %% straightforward relation between the user ID/nickname and the hyperlink,
  %% you can use \printinfo directly e.g.
  % \printinfo{\faMastodon}{@username@instace}[https://instance.url/@username]
  %% But if you absolutely want to create new dedicated info fields for
  %% such platforms, then use \NewInfoField* with a star:
  % \NewInfoField*{mastodon}{\faMastodon}
  %% then you can use \mastodon, with TWO arguments where the 2nd argument is
  %% the full hyperlink.
  % \mastodon{@username@instance}{https://instance.url/@username}
}

\makecvheader
%% Depending on your tastes, you may want to make fonts of itemize environments slightly smaller
% \AtBeginEnvironment{itemize}{\small}

%% Set the left/right column width ratio to 6:4.
\columnratio{0.6}

% Start a 2-column paracol. Both the left and right columns will automatically
% break across pages if things get too long.
\begin{paracol}{2}
\cvsection{Experience}

\cvevent{Junior Software Developer}{Shape Digital}{Since JUL 2023}{São Paulo, SP}
\begin{itemize}
\item Development (implementation of new features, bug fixes, and code refactoring) of the backend API for the DBMS, a software for managing security barriers on ships (Python)
\end{itemize}

\divider

\cvevent{Junior Software Developer}{MegaWhat Energy}{JAN 2021 -- JAN 2023}{São Paulo, SP}
\begin{itemize}
\item Development of the company's web platform (PHP-based CMS and backend API, Vue.js frontend)
\item Management and development of a data dashboard project on the web platform
\item Web scraping of data from the energy sector (Java)
\item Development of a scheduler for scripts in the meteorology sector (Python)
\item Development of platfom mobile app (Flutter)
\end{itemize}

\divider

\cvevent{Software Development Intern}{Aveva Group PLC}{FEB 2020 -- JAN 2021}{São Paulo, SP}
\begin{itemize}
\item Development of product Aveva EDGE (C++ e C\#)
\item Non-automated testing of the team's deliveries
\end{itemize}

\cvsection{Education}

\cvevent{Master's in Computer Science}{Universidade de São Paulo}{Since AUG 2023}{}

\divider

\cvevent{Bachelor's in Computer Science}{Universidade de São Paulo}{FEB 2018 -- JUL 2022}{}
Complete specialization tracks: Systems and Artificial Intelligence 

\switchcolumn

\cvsection{About me}

\begin{quote}
``Enthusiastic gamer and technology lover, I believe that each time we dedicate ourselves to programming, we have the opportunity to significantly improve people's lives. I am constantly seeking to make a positive impact, whether it's solving challenges, creating engaging entertainment experiences, or simplifying work for the next programmers who will interact with the developed code.''
\end{quote}

\cvsection{Skills}

\begin{itemize}
\item Code development following best practices and design patterns
\item Experience with automated tests
\item Experience with agile methods
\item Experience with relational databases (MySQL and SQLServer) and non-relational (Redis and MongoDB)
\item Infrastructure notions (Linux, Docker, AWS)
\end{itemize}

\cvsection{Languages}

\cvskill{Advanced English}{4}
\divider

\cvskill{Basic Spanish}{1.5}

%%\cvsection{Most Proud of}

%%\cvachievement{\faTrophy}{Fantastic Achievement}{and some details about it}

%%\cvsection{Strengths}

%%\cvtag{Hard-working}
%%\cvtag{Eye for detail}\\
%%\cvtag{Motivator \& Leader}

%%\cvsection{Publications}

%% Specify your last name(s) and first name(s) as given in the .bib to automatically bold your own name in the publications list.
%% One caveat: You need to write \bibnamedelima where there's a space in your name for this to work properly; or write \bibnamedelimi if you use initials in the .bib
%% You can specify multiple names, especially if you have changed your name or if you need to highlight multiple authors.
%%\mynames{Lim/Lian\bibnamedelima Tze,
%%  Wong/Lian\bibnamedelima Tze,
%%  Lim/Tracy,
%%  Lim/L.\bibnamedelimi T.}
%% MAKE SURE THERE IS NO SPACE AFTER THE FINAL NAME IN YOUR \mynames LIST

%%\nocite{*}

%%\printbibliography[heading=pubtype,title={\printinfo{\faBook}{Books}},type=book]

%%\cvsection{Publications}

%% Specify your last name(s) and first name(s) as given in the .bib to automatically bold your own name in the publications list.
%% One caveat: You need to write \bibnamedelima where there's a space in your name for this to work properly; or write \bibnamedelimi if you use initials in the .bib
%% You can specify multiple names, especially if you have changed your name or if you need to highlight multiple authors.
%%\mynames{Lim/Lian\bibnamedelima Tze,
%%  Wong/Lian\bibnamedelima Tze,
%%  Lim/Tracy,
%%  Lim/L.\bibnamedelimi T.}
%% MAKE SURE THERE IS NO SPACE AFTER THE FINAL NAME IN YOUR \mynames LIST

%%\nocite{*}

%%\printbibliography[heading=pubtype,title={\printinfo{\faBook}{Books}},type=book]

%% Yeah I didn't spend too much time making all the
%% spacing consistent... sorry. Use \smallskip, \medskip,
%% \bigskip, \vspace etc to make adjustments.
\medskip


\end{paracol}


\end{document}
